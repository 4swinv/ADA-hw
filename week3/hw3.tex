\documentclass{article}
\usepackage[utf8]{inputenc}
\usepackage[english]{babel}
\usepackage{physics}
\usepackage{amsmath}
\usepackage[]{amsthm} %lets us use \begin{proof}
\usepackage[]{amssymb} %gives us the character \varnothing

\title{Advanced Data Analysis\\Homework Week 3}
\author{Aswin Vijay}
\date\today
%This information doesn't actually show up on your document unless you use the make title command below

\begin{document}
\maketitle %This command prints the title based on information entered above

%Section and subsection automatically number unless you put the asterisk next to them.
\section*{Homework 3}

Given the linear model $f_\theta(x) = \sum_{j=1}^{b}\theta_j \phi_j(x)$, where
$\{\phi_j(x)\}_{j=1}^{b}$ are the basis functions, the weighted least squares
objective is given by,
\begin{align*}
    \hat{\theta} = \min_{\theta} \frac{1}{2}\sum_{i=1}^{n} \tilde{w}_i\biggl(f_\theta(x_i)-y_i\biggr)^2
\end{align*}
To derive the analytical solution we rewrite the above objective using the weighting
matrix $\tilde{W} = diag(\tilde{w}_1,\dots,\tilde{w}_n)$ in matrix format as shown below.

\begin{align*}
    \hat{\theta} = \min_{\theta} \biggl[\frac{1}{2} (\Phi\theta - \mathbf{y})^T \tilde{W} (\Phi\theta - \mathbf{y}) \biggr]
\end{align*}
where $\Phi$ is the design matrix and $\mathbf{y} = (y_1,\dots,y_n)^T$. Expanding the brackets and then taking the 
derivative w.r.t $\theta$ we get,
\begin{align*}
    \hat{\theta} &= \min_{\theta} \biggl[\frac{1}{2} (\Phi\theta - \mathbf{y})^T \tilde{W} (\Phi\theta - \mathbf{y}) \biggr]\\
                 &= \min_{\theta} \biggl[\frac{1}{2} (\theta^T\Phi^T - \mathbf{y}^T) \tilde{W} (\Phi\theta - \mathbf{y}) \biggr]\\
                 &= \min_{\theta} \biggl[\frac{1}{2} (\theta^T\Phi^T - \mathbf{y}^T) \tilde{W} (\Phi\theta - \mathbf{y}) \biggr]\\
                 &= \min_{\theta} \frac{1}{2}  \biggl[ \theta^T\Phi^T\tilde{W}\Phi\theta-\theta^T\Phi^T\tilde{W}\mathbf{y}-\mathbf{y}^T\tilde{W}\Phi\theta+\mathbf{y}^T\tilde{W}\mathbf{y}\biggr] \ \ Taking\ derivative\ w.r.t\ \theta\\
                 &= \biggl[ \Phi^T\tilde{W}\Phi\theta-\Phi^T\tilde{W}\mathbf{y}\biggr] = 0\\
\end{align*}
Thus we get the required result,
\begin{align*}
    \hat{\theta} &= (\Phi^T\tilde{W}\Phi)^{-1}\Phi^T\tilde{W}\mathbf{y}
\end{align*}
\end{document}
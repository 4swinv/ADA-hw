\documentclass{article}
\usepackage[utf8]{inputenc}
\usepackage[english]{babel}
\usepackage{physics}
\usepackage{amsmath}
\usepackage[]{amsthm} %lets us use \begin{proof}
\usepackage[]{amssymb} %gives us the character \varnothing

\title{Advanced Data Analysis\\Homework Week 5}
\author{Aswin Vijay}
\date\today
%This information doesn't actually show up on your document unless you use the make title command below

\begin{document}
\maketitle %This command prints the title based on information entered above

%Section and subsection automatically number unless you put the asterisk next to them.
\section*{Homework 5}

Given the complimentary slackness conditions of SVM,
\begin{align}
    \alpha_i(y_i w^T x_i - 1 + \xi_i) = 0\\
    \beta_i \xi_i = 0
\end{align}
and from the dual problem,
\begin{align}
    y_i w^T x_i - 1 + \xi_i &\geq 0\\
    \xi_i &\geq 0\\
    \alpha_i + \beta_i &= C
\end{align}

To prove:
\subsubsection*{1) $\alpha_i=0\implies y_i w^T x_i \geq 1$}
\begin{align*}
    &\alpha_i = 0\\
    &\implies \beta_i = C\ (Eq.\ 5)\\
    &\implies \xi_i = 0\ (Eq.\ 2)\\
    &\implies y_i w^T x_i - 1 \geq 0\ (Eq.\ 3)\\
    &\implies y_i w^T x_i \geq 1
\end{align*}
\subsubsection*{2) $0 < \alpha_i< C\implies y_i w^T x_i = 1$}
\begin{align*}
    0 < \alpha_i < C\\
    &\implies y_i w^T x_i - 1 + \xi_i = 0\ (Eq.\ 1)\\ 
    &\implies 0 < C - \beta_i < C\ (Eq.\ 5)\\
    &\implies 0 < C - \beta_i < C\\
    &\implies C > \beta_i > 0\\
    &\implies \xi_i = 0\ (Eq.\ 2)\\
    &\implies  y_i w^T x_i = 1\ (From\ above)
\end{align*}
\subsubsection*{3) $\alpha_i = C\implies y_i w^T x_i \leq 1$}
\begin{align*}
    &\alpha_i = C\\
    &\implies \beta_i = 0\ (Eq.\ 5)\\
    &\implies  y_i w^T x_i - 1 + \xi_i = 0\ (Eq.\ 1)\\
    &\implies  y_i w^T x_i = 1 - \xi_i\\
    &\implies  y_i w^T x_i \leq 1 \ (Eq.\ 4)
\end{align*}
\subsubsection*{4) $y_i w^T x_i > 1 \implies \alpha_i = 0$}
\begin{align*}
    &y_i w^T x_i - 1>0\ Given\\
    &\implies y_i w^T x_i - 1 + \xi_1>0\ (Eq.\ 4)\\
    &\implies  \alpha_i = 0\ (Eq.\ 1)
\end{align*}
\subsubsection*{5) $y_i w^T x_i < 1 \implies \alpha_i = C$}
\begin{align*}
    &y_i w^T x_i - 1<0\ Given\\
    &but,\ y_i w^T x_i - 1 + \xi_i \geq 0\ (Eq.\ 3)\\
    &\implies \xi_i > 0\ (\xi_i\ can't\ be\ 0)\\
    &\implies \beta_i = 0\ (Eq.\ 2)\\
    &\implies \alpha_i = C\ (Eq.\ 5)
\end{align*}
Thus proven.
\end{document}